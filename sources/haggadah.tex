\documentclass[a4paper,12pt,openany]{memoir}

%\usepackage{xltxtra}
%\defaultfontfeatures{Scale=MatchLowercase,Mapping=tex-text}
\usepackage{charter}
%\usepackage[urw-garamond]{mathdesign}
%\usepackage[bitstream-charter]{mathdesign}
%\usepackage[tpagella]{mathdesign}
%\usepackage{ucs}
%\usepackage[utf8x]{inputenc}  % adding the UTF-8 encoding
%\usepackage{culmus}
%\usepackage[HE8,OT1]{fontenc}
%\usepackage[english,hebrew]{babel}
\usepackage{setspace}
\usepackage{multicol}

\usepackage{polyglossia}
\usepackage{fontspec}

\setdefaultlanguage{english}

\setotherlanguage[numerals=hebrew]{hebrew}

\fontspec [Path = /Users/scott/Library/Fonts/culmus-0.132/ ,UprightFont = *Regular,  BoldFont      = *Demi-Bold]{Shofar}

\newfontfamily\hebrewfont[Script=Hebrew]{Shofar}

\setlength{\beforechapskip}{0cm}
%\setlength{\beforechapskip}{-6.5em}
%\setlength{\topskip}{6.5em}

\newcommand{\HgInst}[1]{{\noindent\sffamily{\bfseries{#1}}}}
\newcommand{\HgEllipsis}{\ensuremath{\left[\ldots\right]}}
\newcommand{\HgSource}[1]{\hfill{\small---\itshape{#1}}}
\newcommand{\hchapter}[1]{
  \begin{hebrew}
    \begin{Spacing}{.97}
      \newpage
      \strut

      \vspace{.15em}

      \noindent\Huge #1

      \vspace{1em}
    \end{Spacing}
  \end{hebrew}
  %\setdefaultlanguage{hebrew}
  %\begin{hebrew}[numerals=hebrew]
  %  \chapter*{#1}
  %%\chapter{Hello}
  %\end{hebrew}
  %  %\chapter*{#1}
  %%\setdefaultlanguage{english}
}
\newcommand{\HgHL}[1]{{\Large\textbf{#1}\par\noindent\\[-.5em]}}
\newcommand{\HgFill}{\vfill \hrule \vfill}

\newenvironment{HgEnglish}{\strut\\\noindent}{\vspace{1em}}
\newenvironment{HgTranslit}{\strut\\\noindent\begin{itshape}}{\end{itshape}\vspace{1em}}
\newenvironment{HgHebrew}{\begin{hebrew}\strut\\\noindent\Large}{\end{hebrew}}
%\newenvironment{HgHebrew}{\begin{otherlanguage}{hebrew}\strut\\\noindent\Large
%}{\par\end{otherlanguage}\vspace{1em}}

%\newcommand{\CSrc}{\textsuperscript{\upshape{[C]}}}
\newcommand{\CSrc}{}
\newcommand{\JSrc}{\textsuperscript{\upshape{[J]}}}
\newcommand{\LSrc}{\textsuperscript{\upshape{[L]}}}
\newcommand{\SSrc}{\textsuperscript{\upshape{[S]}}}

\begin{document}

%\selectlanguage{english}
\aliaspagestyle{chapter}{empty}
\pagestyle{empty}
\strut
\vfill

\begin{center}
\fontsize{22pt}{22pt}
\selectfont
The 
Columbia
Commune
Churchill \\
\fontsize{50pt}{50pt}
\selectfont
Free 
Haggadah
\end{center}
\vspace{10em}
\vfill

%\chapter*{Prefatory miscellany}
%
%\begin{itemize}
%  \item The letter \d{h} (that's ``h-with-a-dot-under-it'') is a voiceless
%    uvular fricative, as in the Hebrew ``hanukkah'', the Scottish ``loch'', or
%    the sound one make while clearing one's throat of mucus.
%
%  \item All translations of scripture are from the 1917 JPS Version unless
%    otherwise noted.
%\end{itemize}

\chapter*{The Passover seder}

\vfill

\vspace{-2em}
\begin{HgHebrew}
  \begin{center}
  קדש 
  -
  ורחץ
  -
  כרפס 
  -
  יחץ 
  -
  מגיד 
  -
  רחצה 
  -
  מוציא
  -
  מצה 
  \\
  מרור 
  -
  כורך 
  -
  שולחן עורך 
  -
  צפון
  -
  ברך 
  -
  הלל 
  -
  נירצה 
  \end{center}
\end{HgHebrew}
\vspace{-3em}
\begin{HgTranslit}
  \begin{center}
  {\small 
    KADESH - UR\d{H}ATZ - KARPAS - YA\d{H}ATZ - % \\
    MAGID - RA\d{H}TZA - MOTZI - MATZAH \\ 
    MAROR - KOREI\d{H} - SHUL\d{H}AN OREI\d{H} - % \\
    TZAFUN - BAREI\d{H} - HALLEL - NIRTZAH}
  \end{center}
\end{HgTranslit}

\vfill

\hchapter{סדר}

We are gathered tonight to retell the story of the exodus from Egypt. ``Seder''
means ``order'', and ``haggadah'' means ``the telling''. This meal is at once an
opportunity to recall our history, a ritual substitute for the Passover
sacrifice that was offered in the Temple days, and a celebration of the freedom
we now enjoy.

\HgFill

\settowidth{\versewidth}{matzah balls in salted water or broth, because you can.}
\begin{verse}[\versewidth]
  Breakfast on kosher macaroons and Diet Pepsi\\
  in the car on the way to Price Chopper for lamb.

  Peel five pounds of onions and let the Cuisinart\\
  shred them while you push them down and weep.

  Call your mother because you know she’s preparing\\
  too, because you want to ask again whether she cooks

  matzah balls in salted water or broth, because you can.\\
  Crumble boullion cubes like clumps of wet sand.

  Remember the precise mixing order, beating \\
  then stirring then folding, so that for one moment

  you can become your grandfather. \\
  Remember the year he taught you this trick,

  not the year his wife died scant weeks before seder\\
  and he was already befuddled when you came home.

  Realize that no matter how many you buy\\
  there are never quite enough eggs at Pesach

  especially if you need twelve for the kugel \\
  and  eighteen for the kneidlach and another dozen

  to hardboil and dip in bowls of stylized tears.\\
  Know you are free! What loss. What rejoicing.
\end{verse}

\HgSource{R. Rachel Barenblat, ``Order''}
\newpage

\chapter*{Kadesh: Sanctification}
\vfill

                                              
                                               
                                                          
\begin{HgHebrew}                                                                            
                                               
                              וַיְהִי עֶרֶב וַיְהִי בֹקֶר יוֹם הַשִּׁשִּׁי׃
וַיְכֻלּוּ הַשָּׁמַיִם וְהָאָרֶץ וְכָל צְבָאָם׃
וַיְכַל אֱלֹהִים בַּיוֹם הַשְּׁבִיעי מְלַאכְתּוֹ אֲשֶׁר עָשָׂה
וַיִשְׁבֹּת בַּיּוֹם הַשְּׁבִיעי מִכָּל מְלַאכְתּוֹ אֲשֶׁר עָשָׂה׃
וַיְבָרֶךְ אֱלֹהִים אֶת יוֹם הַשְּׁבִיעי וַיְקַדֵּשׁ אֹתוֹ
כִּי בוֹ שָׁבַת מִכָּל מְלַאכְתּוֹ אֲשֶׁר בָּרָא אֱלֹהִים לַעֲשׂוֹת׃
סַבְרִי מָרָנָן וְרַבָּנָן וְרַבּוֹתַי!
בָּרוּךְ אַתָּה יְיָ אֱלֹהֵינוּ מֶלֶךְ הָעוֹלָם
בּוֹרֵא פְּרִי הַגָּפֶן.
בָּרוּךְ אַתָּה יְיָ אֱלֹהֵינוּ מֶלֶךְ הָעוֹלָם,
אֲשֶׁר קִדְּשָׁנוּ בְּמִצְוֹתָיו וְרַָצָה בָנוּ,
וְשַׁבָּת קָדְשׁוֹ בְּאַהֲבָה וּבְרָצוֹן הִנְחִילָנוּ,
זִכָּרוֹן לְמַעֲשֵׂה בְרֵאשִׁית.
כִּי הוּא יוֹם תְּחִלָּה לְמִקְרָאֵי קֹדֶשׁ
זֵכֶר לִיצִיאַת מִצְרָיִם.
כִּי בָנוּ בָחַרְתָּ וְאוֹתָנוּ קִדַּשְׁתָּ מִכָּל הָעַמִּים
וְשַׁבָּת קָדְשְׁךָ בְּאַהֲבָה וּבְרָצוֹן הִנְחַלְתָּנוּ.
                            
                                                                                                                  
                                                          
                              \end{HgHebrew}

\begin{HgHebrew}
בָרוּךְ אַתָה יְיָ אֱלֹהֵינוּ מֶלֶךְ הָעוֹלָם שֶהֶחֱיָנוּ וְקִיְמָנוּ

וְהִגִּיעָנוּ לַזְמַן הַזֶה


\end{HgHebrew}
\HgInst{Pour the first cup of wine, and read:}

%And the evening and the morning were the sixth day. 
%Thus the heavens and the earth were finished, and all the host of them. 
%And on the seventh day God ended his work which he had made; 
%And he rested on the seventh day from all his work which he had made. 
%And God blessed the seventh day, and sanctified it: 
%Because that in it he had rested from all his work which God created and made. 

\begin{HgHebrew} 
  בָּרוּךְ אַתָּה יי אֱלֹהֵינוּ מֶלֶךְ הָעוֹלָם 
  \\
  בּוֹרֵא פְּרִי הַגָפֶן.
% ברוך אתה יי אלהינו מלך העולם 
%  \\
%  בורא פרי הגפן.
\end{HgHebrew}

\begin{HgTranslit}
  \HgHL{Baru\d{h} ata Adonai, eloheinu mele\d{h} ha'olam, \\
  borei p'ri hagafen.}
  %Baru\d{h} ata Adonai, eloheinu mele\d{h} ha'olam, \\
  %asher ba\d{h}ar banu mikol am v'romemanu, mikol lashon v'kidshanu, \\
  %b'mitzvotav. \\
  %\HgEllipsis \\
  %Baru\d{h} ata Adonai, m'kadesh Yisrael v'hazmanim.
\end{HgTranslit}
\vspace{-1em}
\begin{HgEnglish}
  \HgHL{Blessed are you, Lord our God, ruler of the universe, \\
  Who creates the fruit of the vine.}
  % Blessed are you, Lord our God, ruler of the universe, \\
  % Who chose us from the throng of people and singled us out among nations \\
  % By giving us the commandments, knowledge of life and good. \\
  % You made festivals for happy times, \\
  % and appointed holidays and seasons for rejoicing. \\
  % Such is this Day of the Feast of Unleavened Bread, when we gather \\
  % To remember our going out from Egypt, and to taste our freedom. \\
  % For you chose us among all others to celebrate \\
  % Your holy festivals with joy and fervor, the marks of your love and favor. \\
  % Blessed are you, God of Israel, who makes holy festivals.

  %\HgSource{adapted from the Red Sea Haggadah}
\end{HgEnglish}

\HgInst{Drink the first cup.}

\vfill

\hchapter{קדש}

\begin{HgEnglish}
  \HgHL{
  Wherefore say unto the children of Israel: I am God, and I will bring you out
  from under the burdens of the Egyptians,}\\[-3em]

  \HgSource{Exodus 6:6 \JSrc}\\

  \noindent There are four expressions of redemption [in Exodus 6:6]: {\itshape
  I will bring you out}, {\itshape I will deliver you}, {\itshape I will redeem
  you} and {\itshape I will take you}. These correspond to the four decrees
  which Pharaoh issued regarding [the Jews]. The sages accordingly ordained four
  cups to be drunk on the eve of Passover to correspond with these four
  expressions, in order to fulfil the verse: {\itshape I will lift up the cup of
  salvation and call upon the name of the Lord.}

  \HgSource{Shemot Rabbah\footnote{Tr. Soncino}}
\end{HgEnglish}

\HgFill

\begin{HgEnglish}
  We rededicate ourselves to liberation from tyranny: from the tyranny of
  poverty, the tyranny of war, the tyranny of ignorance and the tyranny of hate.
  And we light candles to shine as a beacon of liberation for our people and for
  all people. May the light of the candles we kindle tonight bring radiance to
  all those who live in darkness still; may this season, marking our deliverance
  of our people from servitude to Pharaoh, rouse us against anyone who keeps
  others in servitude; may we strive to bring about our own liberation and the
  liberation of all people everyone.

  Blessed is the spirit of freedom in whose honor we kindle the lights of this
  holiday, Passover, the season of freedom. Blessed is the force of life that
  brings us to this year’s spring, and to this renewal of our quest for freedom.
  \SSrc
\end{HgEnglish}

\HgFill

\begin{verse}
  Blessed is the match consumed in kindling flame. \\
  Blessed is the flame that burns in the secret fastness of the heart. \\
  Blessed is the heart with strength to stop its beating for honor's sake. \\
  Blessed is the match consumed in kindling flame.
\end{verse}
\HgSource{Hannah Szenes, ``Blessed is the Match''}

\newpage

\chapter*{Ur\d{h}atz: Washing the hands}

\vfill

\HgInst{Wash your hands (without reciting a blessing).}

\vfill

\hchapter{ורחץ}
\vfill

We wash our hands to prepare ourselves for the Passover rituals.\CSrc

\HgFill

\begin{verse}
Isn’t it always the heart that wants to wash \\
the elephant, begging the body to do it \\
with soap and water, a ladder, hands, \\
in tree shade big enough for the vast savannas \\
of your sadness, the strangler fig of your guilt, \\
the cratered full moon’s light fuelling \\
the windy spooling memory of elephant?
 
What if Father Quinn had said, “Of course you’ll recognize \\
your parents in Heaven,” instead of \\
“Being one with God will make your mother and father \\
pointless.” That was back when I was young enough \\
to love them absolutely though still fear for their place \\
in Heaven, imagining their souls like sponges full \\
of something resembling street water after rain.

\HgEllipsis
\end{verse}
\HgSource{Barbara Ras, ``Washing the Elephant''}
\vfill

\chapter*{Karpas: The green vegetable}

\vfill

\HgInst{Distribute pieces of parsely, and dip them in saltwater.}

\begin{HgHebrew}
  בָּרוּךְ אַתָּה יי אֱלֹהֵינוּ מֶלֶךְ הָעוֹלָם, 
  \\
  בּוֹרֵא פְּרִי הָאֲדָמָה.

%ברוך אתה יי אלהינו מלך העולם, 
%  \\
%  בורא פרי האדמה.
\end{HgHebrew}

\begin{HgTranslit}
  \HgHL{
  Baru\d{h} ata Adonai, eloheinu mele\d{h} ha'olam, \\
  borei p'ri ha'adamah.
}
\end{HgTranslit}
\vspace{-1em}
\begin{HgEnglish}
  \HgHL{
  Blessed are you, Lord our God, ruler of the universe, \\
  Who creates the fruit of the earth.
  }
\end{HgEnglish}

\HgInst{Eat the parsely.}

\vfill

\hchapter{כרפס}

\begin{HgEnglish}
The green vegetable represents spring, and rebirth, renewal and growth. The salt
water represents the tears we shed when we were slaves in Egypt.
\end{HgEnglish}

\HgFill

\begin{HgEnglish}
  The passover seder as we know it, however, actually originated in late
  antiquity, in Roman Palestine in the first centuries of the Common Era. That
  historical context is very important because both the very structure and
  format of the haggadah text reflects not only those conventions but also the
  inevitable cultural changes that occurred when Jews moved from Roman Palestine
  to other cultures in the Diaspora \HgEllipsis

  Let me give one brief example: The passage that we know as the
  Four Questions originally consisted of three, not four questions, and each
  question directly addressed one of the three symbolic foods eaten at the
  seder---the maror (which are the bitter herbs customarily eaten with
  \d{h}aroset) the matzah, and the Passover sacrifice. As it happens, each of
  these three foods also related to the three courses that were customarily
  eaten at a Greco-Roman banquet. \\

  \HgSource{Joseph Tabory, The JPS Commentary on the Haggadah}
\end{HgEnglish}

\HgFill

\begin{HgEnglish}
  Blessed are you, underpaid migrant farmworkers, who create the fruit of the
  earth.\SSrc
\end{HgEnglish}

\HgFill

\begin{verse}
  For, lo, the winter is past, the rain is over and gone;

  The flowers appear on the earth; the time of singing is come, and the voice of
  the turtle is heard in our land;

  The fig-tree putteth forth her green figs, and the vines in blossom give forth
  their fragrance. Arise, my love, my fair one, and come away.
\end{verse}
\HgSource{The Song of Songs 2:11-13 \JSrc}

\begin{HgEnglish}
\end{HgEnglish}

\chapter*{Ya\d{h}atz: Breaking the matzah}

\vfill

\HgInst{Break the middle matzah, and set aside the larger piece as the
afikoman. Hold the remaining matzah up, and read:}

\begin{HgHebrew}
  הָא לַחְמָא עַנְיָא דִּי אֲכָלוּ אַבְהָתָנָא
  \\
  בְּאַרְעָא דְמִצְרָיִם.
  \\
  כָּל דִּכְפִין יֵיתֵי וְיֵכוֹל,
  \\
  כָּל דִּצְרִיךְ יֵיתֵי וְיִפְסַח. 
  \\
  הָשַּׁתָּא הָכָא,
  לְשָׁנָה הַבָּאָה בְּאַרְעָא דְיִשְׂרָאֵל.
  \\
  הָשַּׁתָּא עַבְדֵי,
  לְשָׁנָה הַבָּאָה בְּנֵי חוֹרִין.


% הא לחמא עניא די אכלו אבהתנא
%  \\
%  בארעא דמצריִם.
%  \\
%  כל דכפין ייתי ויכול,
%  \\
%  כל דצריך ייתי ויִפסח. 
%  \\
%  השתא הכא,
%  לשנה הבאה בארעא דיִשראל.
%  \\
%  השתא עבדי,
%  לשנה הבאה בני חורין.
\end{HgHebrew}

\begin{HgTranslit}
  \HgHL{
  Hala\d{h}ma anya di a\d{h}alu avhatana \\
  b'ara d'mitzrayim.
}
  Kol di\d{h}fin yeitei v'ye\d{h}ol \\
  kol ditzri\d{h} yeitei v'yifsa\d{h}. \\
  Hashata ha\d{h}a, %\\
  l'shanah haba'ah b'ara d'yisrael. \\
  Hashata avdei, %\\
  l'shanah haba'ah b'nei \d{h}orin.
\end{HgTranslit}

\begin{HgEnglish}
  \HgHL{
  This is the bread of affliction that our ancestors ate \\
  in the land of Egypt.
  }
  All who are hungry, come and eat; \\
  all who are needy, come and celebrate the Passover. \\
  Now we are here;
  next year may we be in Israel; \\
  now we are slaves;
  next year may we be free.
\end{HgEnglish}

\HgInst{Pour (but not drink!) the second cup of wine.}

\vfill

\hchapter{יחץ}

\begin{HgEnglish}
  According to one tradition, we recite this portion in Aramaic rather than
  Hebrew so that the angels (who can't understand Aramaic) don't take us up on
  our offer and crash the party.

  Various reasons are given for breaking the middle matzah: to remind us that
  the task of liberation, like the matzah, is incomplete; or in commemoration of
  the time when we were poor and hungry. Another commentary tells us that the
  Hebrew slaves, making their way to freedom, couldn't know when their next meal
  would come, so they ate only a small amount of matzah and hid the rest for
  later.
\end{HgEnglish}

\HgFill

\begin{HgEnglish}
  It is written [Deut. xvi. 3]: ``Bread of affliction'' (le\d{h}em oni), and as
  ``oni'' can also stand for ``proclaiming,'' the bread may be called ``bread of
  proclamation,'' i.e., ``bread over which proclamations should be made,'' and
  thus we have also learned [from the Oral Law]. Or ``Oni'' may still be called
  ``poor,'' and for the reason that the benediction pertaining to the eating of
  the unleavened bread should be made over a broken piece after the manner of
  the poor.

  \HgSource{Tractate Pesachim\footnote{Tr. Robinson}}
\end{HgEnglish}

\HgFill

\begin{HgEnglish}
  While there is a lower class, I am in it; while there is a criminal element, I
  am of it; while there is a soul in prison, I am not free!

  \HgSource{Eugene V. Debs}
\end{HgEnglish}

\HgFill

\begin{HgEnglish}
This invitation appears to be presented at the wrong time, for Kiddush has
already been recited, in the wrong place, for it is issued in the privacy of our
own homes, and in a language, Aramaic, which most people no longer understand.
[\ldots] What purpose do these invitations serve?

The Pesach Seder is a celebration of our redemption and we are all guests of
honor. To prevent the guests from feeling beholden to [host,] which would
inhibit their involvement and participation in the evening, we begin the Seder
by allowing the guests to invite others. The Talmud states "a guest is not
permitted to invite other guests". However, a guest of honor has the right to
invite whomever he chooses. The message we are relaying to all the participants
is they are not merely guests beholden to the homeowner. Rather, they are all
guests of honor, celebrating their own redemption. [\ldots] The purpose of the
invitation is for the guests already assembled, not for those who are absent.

\HgSource{R. Yochanan Zwieg, ``Ha Lachma Anya''}
\end{HgEnglish}

\chapter*{Magid: The Passover story \\ {\LARGE The four questions}}

\vfill

\HgInst{The youngest person present recites:}

\begin{HgHebrew}
מַה נִּשְּׁתַּנָה הַלַּיְלָה הַזֶּה מִכָּל הַלֵּילוֹת? 
\\
שֶׁבְּכָל הַלֵּילוֹת אָנוּ אוֹכְלִין חָמֵץ וּמַצָּה,
%\\
הַלַּיְלָה הַזֶּה כּוּלוֹ מַצָּה. 
\\
שֶׁבְּכָל הַלֵּילוֹת אָנוּ אוֹכְלִין שְׁאָר יְרָקוֹת,
%\\
הַלַּיְלָה הַזֶּה מָרוֹר. 
\\
שֶׁבְּכָל הַלֵּילוֹת אֵין אֶנוּ מַטְבִּילִין אֲפִילוּ פַּעַם אֶחָת, 
%\\
הַלַּיְלָה הַזֶּה שְׁתֵּי פְעָמִים. 
\\
שֶׁבְּכָל הַלֵּילוֹת אָנוּ אוֹכְלִין בֵּין יוֹשְׁבִין וּבֵין מְסֻבִּין, 
%\\
הַלַּיְלָה הַזֶּה כֻּלָנו מְסֻבִּין. 
\end{HgHebrew}

\begin{HgTranslit}
  \HgHL{Ma nishtanah halailah hazeh mikol haleilot?}
  Shebe\d{h}ol haleilot, anu o\d{h}lin \d{h}ametz umatzah, 
  halailah hazeh kulo matzah. \\
  Shebe\d{h}ol haleilot, anu o\d{h}lin sh'ar y'rakot, 
  halailah hazeh maror. \\
  Shebe\d{h}ol haleilot, ein anu matbilin afilu p'am e\d{h}ad,
  halailah hazeh sh'tei f'amin. \\
  Shebe\d{h}ol haleilot, anu o\d{h}lin bein yoshvin uvein m'subin,
  halailah hazeh \\ \strut $\quad$ kulanu m'subin.
\end{HgTranslit}

\begin{HgEnglish}
  \HgHL{Why is this night different from all other nights?}
  On other nights, we eat leavened bread and matzah; tonight only matzah. \\
  On other nights, we eat all kinds of herbs; tonight bitter herbs. \\
  On other nights, we do not dip our food; tonight we dip twice. \\
  On other nights, we eat upright or reclining; tonight we all recline. \\
\end{HgEnglish}

\HgInst{Uncover the matzah, and read:}

\begin{HgEnglish}
\HgHL{We were slaves of Pharaoh in Egypt, and our God brought us out from there
with a mighty hand and an outstretched arm.}
Now, if God had not brought our ancestors out from Egypt, then we, our
children, and our children's children might still be enslaved to Pharaoh in
Egypt.  Therefore, even if we were all wise, all understanding, all learned in
the ways of Torah, we would still be obligated to tell the story of the Exodus.
And indeed, everyone who dwells upon the features of the Exodus is
praiseworthy.
\end{HgEnglish}

\vfill

\hchapter{מגיד}

\vfill


\begin{HgEnglish}
It happened that Rabbi Eliezer, Rabbi Yehoshua, Rabbi Elazar ben Azaryah, Rabbi
Akiva and Rabbi Tarphon were reclining [at a seder] in B'nei Berak. They were
discussing the exodus from Egypt all that night, until their students came and
told them: ``Masters! The time has come for reciting the morning Sh'ma!''

Rabbi Eleazar ben Azariah said: I have lived to be a man of threescore years and
ten, yet I did not understand why the story of the Exodus should be told at
night until Ben Zoma explained it to me. He said: It is said, ``That thou mayest
remember the day when thou camest forth out of the land of Egypt all the days of
thy life.'' (Deuteronomy 16:3) ``The days of your life'' would have meant the
days only, but ``all the days of your life'' includes the nights also.  The
Sages of Israel explain it further: ``The days of your life'' refers to this
world, while ``All the days of your life'' includes the time of the Messiah.
\LSrc
\end{HgEnglish}

\HgFill

\begin{enumerate}
  \item
    Why do we eat the bread of our neighbors, instead of inviting them over to
    break matzo with us and figuring out a way to give each other a place at the
    table in a free Middle East?

  \item
    Why do we swallow the bitter herbs of ethnocentrism, ultranationalism, and
    pharaoic fundamentalism [?]

  \item
    Why do we recline, most nights of the year, when our freedom remains
    incomplete, so long as the "giant triplets of racism, extreme materialism,
    and militarism" are still at large in these lands?

  \item
    Why do we dip into the fountain of forgetting more than we dip into the
    \linebreak
    charoset and the salt water of memory, which are meant to remind us not to
    do unto others as has been done unto us?
\end{enumerate}
\HgSource{Michael Gould-Wartofsky, ``Liner Notes for a Freedom Seder''}\\

\HgFill

\begin{HgEnglish}
  The Seder is thus a giant yet intimate classroom, with the Haggada serving as
  the subject matter, and the father as the main educator entrusted with the
  role of engaging pupils in a discussion based on a question and answer format
  [\ldots] in the hope of involving everyone present. Listening is passive and
  stagnant; far better to inquire, probe and analyze our oldest Jewish festival.

  \HgSource{Joe Bobker, ``The Exodus Story and its Message''}
\end{HgEnglish}

\vfill

\chapter*{Magid: The Passover story \\ {\LARGE The four children}}

\vfill

\HgHL{The Torah speaks of four kinds of children: The wise child, the wicked
  child, the simple child, and the child who does not know how to ask.}\\
The wise child asks: ``What is the meaning of the testimonies, laws and
judgments which God has commanded you?''

To that one, you explain all the laws of Passover, down to the very last detail
about the Afikoman. \\[1em]
The wicked child asks: ``What does all of this mean to you?'' (Exodus 12:26)

By saying ``you,'' and not ``we'' or ``me,'' he excludes himself from the group,
and denies God. Answer that child plainly: ``This is done because of what the
Lord did for me when I came out of Egypt.'' (Exodus 13:8) ``For me, not for you:
if you had been there in Egypt, you would not have been redeemed.''\\[1em]
The simple child asks: ``What is this?'' 

Answer that one: ``By strength of hand the Lord brought us out from Egypt, from
the house of bondage.'' (Exodus 13:14)\\[1em]
To the child who does not know how to ask, speak first, it is written: ``And
thou shalt show thy son in that day, saying, This is done because of that which
the Lord did unto me when I came forth out of Egypt.'' (Exodus 13:8) \LSrc

\vfill

\hchapter{מגיד}

\begin{HgEnglish}
Wise, wicked, simple, does not know how to ask. This order conforms neither to
the order in which the children are mentioned in the Torah (Wicked, Simple, Does
Not Know How to Ask, Wise), nor to the order of their moral standing, in which
the wicked child should be last.

Rather, they are listed in order of their intellectual capacities: The wise
child; the wicked child, who is also wise but whose insolence leads him to act
wickedly; the simple child, who has at least enough intelligence to ask; and
finally the one who does not know how to ask

\HgSource{Chabad}
\end{HgEnglish}

\HgFill

\begin{HgEnglish}
  When the wicked child says “What is this service to you?” we assume he is
  excluding himself. But doesn’t the wise child also refer to “the
  testimonies\ldots{}that God our God has commanded to you”?

  The wicked child makes no mention of God; the wise child refers to “God our
  God,” clearly including himself. He uses “you” in the sense of “you who came
  out of Egypt and received God’s commandments,” as opposed to himself who was
  not yet born when the commandment was given.

  \HgSource{Ma\d{h}zor Vitri \LSrc}
\end{HgEnglish}

\HgFill

\begin{HgEnglish}
  While the Four Sons differ from one another in their reaction to the Seder,
  they have one thing in common: they are all present at the Seder.
  \HgEllipsis{} Unfortunately, in our time of confusion and spiritual
  bankruptcy, there is another kind of a Jewish child---a ``fifth son'', who is
  conspicuous by his absence from the Seder; the one who has no interest
  whatsoever in Torah and commandments, laws and customs; who is not even aware
  of the Passover Seder, of the Exodus from Egypt and the subsequent Revelation
  at Sinai. \HgEllipsis

  Finding themselves a small minority, and encountering social and economic
  difficulties, some parents had the mistaken notion, which they transmitted to
  their children, that the way to overcome these difficulties is to become
  quickly assimilated into the new environment by discarding the heritage of
  their forefathers and abandoning the Jewish way of life. Finding that this
  process leads to the discomfort of inner spiritual conflict, some parents
  resolved to spare their children this conflict altogether. \HgEllipsis

  By this attitude, these parents hoped to assure their children's existence and
  survivial in the new environment. But what kind of existence is it, if
  everything spiritual and holy is traded for the material? What kind of
  survival is it, if it means the sacrifice of the soul for the amenities of the
  body?

  \HgSource{R. Menachem Mendel Schneerson}
\end{HgEnglish}


\chapter*{Magid: The Passover story \\ {\LARGE Exile, Bondage and Deliverance}}

\vfill

\HgInst{If many people present don't know the full Passover story, consider
reading the facing page instead.}

\begin{HgEnglish}
Go and hear what Laban the Aramaean wanted to do to our father Jacob. Pharaoh
had issued a decree against the male children only, but Laban wanted to uproot
everyone. As it is said: 
\HgHL{%
  ``A wandering Aramean was my father, and he went
  down into Egypt, and sojourned there, few in number; and he became there a
  nation, great, mighty, and populous. And the Egyptians dealt ill with us, and
  afflicted us, and laid upon us hard bondage.  And we cried unto the Lord, the
  God of our fathers, and he heard our voice, and saw our affliction, and our
toil, and our oppression. And God brought us forth out of Egypt with a mighty
hand, and with an outstretched arm, and with great terribleness, and with signs,
and with wonders.
}
\end{HgEnglish}
\vspace{-1em}

\HgFill

\begin{HgEnglish}
  \small
\noindent
``and he went down into Egypt'' forced by divine decree.\\

\noindent
``and sojourned there'' not to settle, but only to live there temporarily, as
it is said: ``They said to Pharaoh, We have come to sojourn in the land, for
there is no pasture for your servants' flocks because the hunger is severe in
the land of Canaan; and now, please, let your servants dwell in the land of
Goshen.''
%\\ 
%\noindent
\HgEllipsis \\

% \noindent
% ``few in number'' as it is said: ``Your fathers went down to Egypt with seventy
% persons, and now, the Lord, your God, has made you as numerous as the stars of
% heaven.''\\
% 
% \noindent
% ``and he became there a nation'' set apart from the Egyptians.\\

\noindent
``great, mighty,'' as it is said: ``And the children of Israel were fruitful and
increased abundantly, and multiplied and became very, very mighty, and the land
became filled with them.''\\

\noindent
``and populous'' as it is said: ``I passed over you and saw you wallowing in
your bloods, and I said to you `By your blood you shall live,' and I said to you
`By your blood you shall live!' I caused you to thrive like the plants of the
field, and you increased and grew and became very beautiful, your bosom
fashioned and your hair grown long, but you were naked and bare.''\LSrc

%\noindent\HgEllipsis

%\noindent
%``And the Egyptians dealt ill with us,'' as it is said: ``Come, let us act
%cunningly with [the people] lest they multiply and, if there should be a war
%against us, they will join our enemies, fight against us and leave the land.''\\
%
%``and afflicted us,'' as it is said: ``They set taskmasters over [the people of
%Israel] to make them suffer with their burdens, and they built storage cities
%for Pharaoh, Pitom and Ramses.''\\
%
%\noindent
%``and laid upon us hard bondage,'' as it is said: ``The Egyptians made the
%children of Israel work with rigor. And they made their lives bitter with hard
%work, with mortar and with bricks and all manner of service in the field, all
%their work which they made them work with rigor.'' \\[2em]
%
%\noindent
%``And God brought us forth out of Egypt,'' not through an angel, not through a
%seraph and not through a messenger. The Holy One, blessed be He, did it in His
%glory by Himself! \HgEllipsis \\
%
%\noindent
%``With a mighty hand,'' this refers to the pestilence, as it is said:
%``Behold, the hand of the Lord will be upon your livestock in the field, upon
%the horses, the donkeys, the camels, the herds and the flocks, a very severe
%pestilence.'' \\
%
%\noindent
%``and with an outstretched arm,'' this refers to the sword, as it is said: ``His
%sword was drawn, in his hand, stretched out over Jerusalem.'' \\
%
%\noindent
%``and with a great terribleness,'' this refers to the revelation of the Divine
%Presence, as it is said: ``Has any God ever tried to take for himself a nation
%from the midst of another nation, with trials, signs and wonders, with war and
%with a strong hand and an outstretched arm, and with great manifestations, like
%all that the Lord your God, did for you in Egypt before your eyes!'' \\
%
%\noindent
%``and with signs,'' this refers to the staff, as it is said: ``Take into your
%hand this staff with which you shall perform the signs.'' \\
%
%\noindent
%``and with wonders,'' this refers to the blood, as it is said: ``And I shall
%show wonders in heaven and on earth.

\end{HgEnglish}

\HgFill

\begin{HgEnglish}
  \small
  I will sing unto God, for he has triumphed greatly. \hfill Horse \\
  and rider he has cast into the sea. \hfill God is my strength and song, and he shall be \\
  my deliverance. \hfill This is my God, and I will glorify him; \hfill the God \\
  of my father, and I will exalt him. \hfill God is a man of war; God \\
  is his name. \hfill He has thrown Pharaoh's chariots and host into the sea,
  \hfill and sunk \\
  his favorite captains in the Sea of Reeds.

\HgSource{Deuteronomy 15:1-4}
\end{HgEnglish}


\hchapter{מגיד}

\vfill

\begin{HgEnglish}
  \small
  There was a famine in Canaan.
  So Jacob,
  his wives and his concubines,
  his sons, their wives,
  their children and their servants went down to Egypt.
  Jacob's people were shepherds, 
  so the Pharaoh gave them land in Goshen for grazing, 
  and they flourished there.

  They had many children. 
  Jacob died, 
  then the Pharaoh, 
  then all of Jacob's children,
  but by that time the Israelites were so numerous
  the new Pharaoh began to fear them.
  So he took away
  their fields and their flocks, 
  and made them slaves.
  He set taskmasters over them,
  and forced them to build the granary-cities of Pitom and Ramses.

  Still they multiplied, 
  and still the Pharaoh was afraid. 
  Finally he decided to destroy them entirely: 
  He decreed that all the male children of the Israelites 
  would be drowned in the Nile.
  
  There was a woman, a Levite,
  who had a son and hid him for three months.
  When she could no longer conceal him,
  she placed him in a basket of bulrushes,
  and floated him on the Nile.
  The Pharaoh's daughter, bathing in the river, found him.
  She said,
  ``This is an Israelite child, 
  but I will raise him as my son.''
  She drew him out of the river, 
  and called him Moses, which means
  ``who is drawn out''.

  Once, when Moses was older, 
  he came across an Egyptian
  beating an Israelite slave. 
  He was so enraged he struck the Egyptian, who fell down dead.
  Fearing punishment, 
  Moses fled to the hills
  and lived there for years with a neighboring tribe.

  One day, walking, he saw smoke. 
  The smoke came from a bush, 
  which burned but did not burn up in the fire. 
  God spoke to Moses from the bush, 
  and commanded him to lead the Israelites out of Egypt. 
  ``Lord,'' said Moses,
  ``I am slow of speech. 
  Take someone else.'' 
  God said ``Then let your brother Aaron speak for you.
  You will put the words in his mouth, 
  and I will put them in yours.
  You will lead the Israelites out of Egypt.''

  So Moses and Aaron went to the Pharaoh, 
  and said ``Let my people go.'' 
  And Aaron threw his staff on the ground, where it became a snake.
  The Egyptian magicians also threw their staffs on the ground,
  and turned them into snakes,
  but Aaron's snake ate the others.
  Then the Pharaoh was afraid. 
  But God hardened the Pharaoh's heart,
  and he would not let the Israelites go.

  Then God sent plagues down on the Egyptians: 
  he turned the Nile to blood, 
  made it crawl with frogs and lice and gnats, 
  afflicted the Egyptians with pestilence and boils, 
  with hail and locusts, 
  and he covered Egypt in darkness.
  After each plague Moses and Aaron went to the Pharaoh 
  and said ``Let my people go,'' 
  and each time God hardened the Pharaoh's heart, 
  and he would not let them go.

  Finally God said to Moses: 
  ``Now you will go free.
  Let each Israelite family ask the Egyptians
  for jewelry of silver and gold: 
  they will give it to you.
  Then let each Israelite family slaughter a lamb, 
  and paint the door-posts of their house with lamb's blood.'' 
  They did. 
  That night, God sent the Angel of Death to Egypt, 
  and the Angel went into each Egyptian house 
  and killed each family's first-born son while he slept. 
  But the Angel passed over the houses that were marked with blood.

  In the night the the Pharaoh awoke
  and saw that his son was dead. 
  He said to Moses and Aaron,
  ``Rise up and go, leave as you have promised,
  and bless me.''
  So the Israelites left in haste,
  carrying Egyptian gold
  and bread that had not yet risen.
  But soon the the Pharaoh changed his mind, 
  and he chased after them with horses and chariots.

  The Israelites came to the shore of the Sea of Reeds, 
  and could not cross. 
  God said to Moses ``Lift up your staff,
  and stretch your hand out over the sea.'' 
  Moses lifted up his staff, 
  and stretched out his hand,
  and the sea parted. 
  The Israelites crossed on dry land. 
  They ran, chased by Egyptian chariots.
  But when the last of the Israelites stepped onto the far shore, 
  the sea closed again.
  The Egyptians drowned, 
  and we were free.

\end{HgEnglish}

%\HgFill
%
%\begin{HgEnglish}
%  The Torah records the song we sang as we crossed the red sea. One of the
%  best-known verses begins ``{\itshape Mi \d{h}amo\d{ha} ba'elim,
%Adonai?}''---``Who is like you Lord, among all the gods?'' But the correct
%Hebrew is {\itshape kamo\d{h}a}, not {\itshape \d{h}amo\d{h}a}. It's said that
%when the Jews first came to the shore the sea didn't part, but one man entered
%the water singing. {\itshape Mi \d{h}amo\d{h}a} is the sound of him beginning to
%choke on the water; only after he was fully submerged did the sea begin to part.
%\end{HgEnglish}

\chapter*{Magid: The Passover story\\ {\LARGE The plagues}}

\vfill

\HgInst{Dip a finger in the cup, and spill a drop of wine on the plate as you
read each plague. Don't lick your finger!}

\noindent
\begin{minipage}{.33\textwidth}
  \vspace{-1.2em}
  \begin{HgTranslit}
    \begin{Spacing}{1.2}
    Dam \\
    Tzfardea \\
    Kinim \\
    Arov \\
    Dever \\
    Sh'\d{h}in \\
    Barad \\
    Arbeh \\
    \d{H}oshe\d{h} \\
    Ma\d{h}at b'\d{h}orot
    \end{Spacing}
  \end{HgTranslit}
\end{minipage}
\begin{minipage}{.33\textwidth}
  \vspace{-1.2em}
  \begin{HgEnglish}
    \begin{Spacing}{1.2}
      Blood \\
      Frogs \\
      Lice \\
      Beasts \\
      Pestilence \\
      Boils \\
      Hail \\
      Locusts \\
      Darkness \\
      Death of the firstborn
    \end{Spacing}
  \end{HgEnglish}
\end{minipage}
\begin{minipage}{.32\textwidth}
  \begin{HgHebrew}
    \begin{Spacing}{1}
      \vspace{-2em}
  דָּם 
  \\
  צְפֵרְדֵּעַ 
  \\
  כִּנִים 
  \\
  עָרוֹב 
  \\
  דֶּבֶר 
  \\
  שְׁחִין 
  \\
  בָּרד
  \\
  אַרְבֶּה 
  \\
  חשֶׁךְ 
  \\
  מַכַּת בְּכוֹרוֹת 
  \end{Spacing}
  \end{HgHebrew}
\end{minipage}

\HgInst{Then read or sing:}

\begin{HgHebrew}
  אִלּוּ הוֹצִיאָנוּ מִמִּצְרָיִם
  \HgEllipsis{}
  דַּיֵּנוּ!
  \\
  אִלּוּ נָתַן לָנוּ אֶת הַשַּׁבָּת
  \HgEllipsis{}
  דַּיֵּנוּ!
  \\
  אִלּוּ נָתַן לָנוּ אֶת הַתּוֹרָה
  \HgEllipsis{}
  דַּיֵּנוּ!
\end{HgHebrew}

\begin{HgTranslit}
  Ilu hotzianu mimitzrayim \HgEllipsis{} dayenu! \\
  Ilu natan lanu et hashabbat \HgEllipsis{} dayenu! \\
  Ilu natan lanu et hatorah \HgEllipsis{} dayenu!
\end{HgTranslit}

\begin{HgEnglish}
  \HgHL{
  Had he only brought us out of Egypt, and not judged them, dayenu \\ (it would
  have been enough)!}
  Had he only judged them, and not their idols, dayenu!\\
  Had he only destroyed their idols, and not their first-born, dayenu!\\
  Had he only destroyed their first-born, and not given us their wealth, dayenu!\\
  %Had he only given us their wealth, and not split the sea for us, dayenu!\\
  %Had he only split the sea for us, and not taken us through on dry land, dayenu!\\
  %Had he only taken us through, and not drowned our oppressors, dayenu!\\
  \HgEllipsis\\
  Had he only given us the sabbath, and not brought us to Mount Sinai, dayenu!\\
  Had he only brought us to Mount Sinai, and not given us the Torah, dayenu!\\
  Had he only given us the Torah, and not brought us into Israel, dayenu!\\
  Had he only given us the Torah, and not built the temple for us, dayenu!
\end{HgEnglish}

\vfill

\hchapter{מגיד}

\begin{HgEnglish}
  We spill our wine so that our pleasure during the holiday is tempered in
  remembrance of the  suffering of the Egyptians.
\end{HgEnglish}

\HgFill

\begin{HgEnglish}
  In that hour the ministering angels wished to sing before the Holy One,
  blessed be he. But he rebuked them, saying: ``The work of my hands is drowning
  in the sea, and you would sing hymns?''
  
  \HgSource{Sanhedrin 39b}
\end{HgEnglish}

\HgFill

\begin{HgEnglish}
We still have a long, long way to go before we reach the promised land of
freedom. Yes, we have left the dusty soils of Egypt, and we have crossed a Red
Sea that had for years been hardened by a long and piercing winter of massive
resistance, but before we reach the majestic shores of the promised land, there
will still be gigantic mountains of opposition ahead and prodigious hilltops of
injustice\ldots{}let us go out with a divine dissatisfaction. \HgEllipsis

Let us be dissatisfied until from every city hall, justice will roll down like
waters, and righteousness like a mighty
stream.

Let us be dissatisfied until that day when the lion and the lamb shall lie down
together, and every man will sit under
his own vine and fig tree, and none shall be afraid.

\HgSource{Martin Luther King, Jr.}
\end{HgEnglish}

\HgFill

\begin{HgEnglish}
Rabbi Yosi the Gallilean said: How do you know that the Egyptians were stricken
by ten plagues in Egypt, and then were struck by fifty plagues at the sea?

In Egypt it says of them, "The magicians said to Pharaoh `This is the finger of
God.' At the sea it says, "Israel saw the great hand that the Lord laid against
Egypt; and the people feared the L-rd, and they believed in the Lord and in his
servant Moses."

Now, how often were they smitten by `the finger'? Ten plagues!

Thus you must conclude that in Egypt they were smitten by ten plagues, at the
sea they were smitten by fifty plagues! \LSrc

\end{HgEnglish}

\chapter*{Magid: The Passover story\\ {\LARGE The seder plate \& second cup}}
\vspace{-2em}

\begin{HgEnglish}
Rabban Gamliel used to say: \HgHL{Whoever does not discuss the following three
things on Passover has not fulfilled his duty: the passover-sacrifice, the
matzah, and the maror.}
\end{HgEnglish}

\HgInst{Hold the shankbone, and ask:}
\begin{HgEnglish}
  This passover lamb that our ancestors ate in the days of the temple: why did
  they eat it?

  Because the Lord passed over our ancestors' houses in Egypt, when he struck
  the Egyptians with a plague.
\end{HgEnglish}

\HgInst{Hold the broken matzah, and ask:}
\begin{HgEnglish}
  This matzah: why do we eat it?

  Because our ancestors' dough did not have time to rise before the Lord
  revealed himself to them, and redeemed them.
\end{HgEnglish}

\HgInst{Hold the bitter herb, and ask:}
\begin{HgEnglish}
  This maror: why do we eat it?

  Because the Egyptians made our ancestors' lives bitter with hard labor, and
  with bricks and mortar.
\end{HgEnglish}\\

\HgInst{Then read:}
\begin{HgEnglish}
  \HgHL{In every generation, let each one say that he himself came out of
  Egypt.} As it is said: ``You shall tell your child on that day, `it is because
  of what God did for me when I left Egypt.''\\
\end{HgEnglish}

\HgInst{Finally, hold up the wine, and read:}

\begin{HgHebrew} 
  בָּרוּךְ אַתָּה יי אֱלֹהֵינוּ מֶלֶךְ הָעוֹלָם 
  \\
  בּוֹרֵא פְּרִי הַגָפֶן.
\end{HgHebrew}

\begin{HgTranslit}
  Baru\d{h} ata Adonai, eloheinu mele\d{h} ha'olam,
  borei p'ri hagafen.
\end{HgTranslit}

\begin{HgEnglish}
  Blessed are you, Lord our God, ruler of the universe,
  who creates the fruit of the vine.
\end{HgEnglish}

\HgInst{Drink the second cup.}
\vfill

\hchapter{מגיד}

\begin{HgEnglish}
  \HgHL{and I will deliver you from their bondage,}

  \vspace{-2em}
  \HgSource{Exodus 6:6 (JPS)}
\end{HgEnglish}

%\HgFill
%
%\begin{HgEnglish}
%  Tradition tells us that when God dispatched the Angel of Death to kill the
%  Egyptians' first-born sons, he instructed the Jews to slaughter a lamb and
%  paint their doorways with its bloody shankbone. The Angel {\itshape passed
%  over} the marked houses, sparing the Jewish children inside.
%\end{HgEnglish}

\HgFill

\strut

\HgInst{Hold up the \d{h}aroset, and ask:}
\begin{HgEnglish}
  This \d{h}aroset, why do we eat it?

  It has two symbolic meanings in the Seder: it reminds us of the hard Jewish
  labor performed with bricks and mortar; and it is sweet to counteract the
  bitterness of the maror, symbolizing the sweetness of freedom that followed
  the bitterness of slavery.
\end{HgEnglish}

\HgInst{Hold up the egg, and ask:}
\begin{HgEnglish}
  This egg, why do we eat it?

  The egg is traditionally a symbol of mourning -- to remind us of the
  destruction of the Temple-- but it is also a symbol of rebirth and new life
  that comes with the spring. We dip it in salt water, like the parsley, to
  remind ourselves of the rebirth of the Jewish people as a free people came
  from our tears in slavery.

  {\itshape Blessed are the chickens whose eggs we steal and eat.} \SSrc
\end{HgEnglish}

\HgFill

\begin{HgEnglish}
  It's often reported that the tradition of placing an orange on the seder plate
  began after an angry theologian declared that there would be an orange on
  every seder plate before women became rabbis. In fact, the orange began as an
  expression of solidarity with lesbians, gay men and other marginalized members
  of the Jewish community, and was itself adapted from a feminist seder at
  Oberlin which put a piece of bread on the seder plate for the same reason.
\end{HgEnglish}

\HgFill

\begin{HgEnglish}
  According to Exodus 12:37-38, the Israelites numbered ``about six hundred
  thousand men on foot, besides women and children,'' plus many non-Israelites
  and livestock. Numbers 1:46 gives a more precise total of 603,550. The
  600,000, plus wives, children, the elderly, and the ``mixed multitude'' of
  non-Israelites would have numbered some 2 million people, compared with an
  entire Egyptian population in 1250 BCE of around 3 to 3.5 million.
  Marching ten abreast, and without accounting for livestock, they would have
  formed a line 150 miles long.

  No evidence has been found that indicates Egypt ever suffered such a
  demographic and economic catastrophe or that the Sinai desert ever hosted (or
  could have hosted) these millions of people and their herds.

  \HgSource{Wikipedia}
\end{HgEnglish}

\chapter*{Ra\d{h}tza: Washing the hands again}

\vfill

\HgInst{Wash your hands, and individually recite:}

\begin{HgHebrew}
בָּרוּךְ אַתָּה יי אֱלֹהֵינוּ מֶלֶךְ הָעוֹלָם,
\\
אֲשֶׁר קִדְשָׁנוּ בְּמִצְוֹתָיו 
\phantom{וְצִוָּנוּ עַל נְטִילַת יָדַיִם.}
\\
וְצִוָּנוּ עַל נְטִילַת יָדַיִם.
\end{HgHebrew}

\begin{HgTranslit}
  Baru\d{h} ata Adonai, eloheinu mele\d{h} ha'olam, \\
  asher kidshanu b'mitzvotav, \\
  v'tzivanu al n'tilat yadayim.
\end{HgTranslit}

\begin{HgEnglish}
  Blessed are you, Lord our God, ruler of the universe, \\
  who sanctifies us with your commandments, \\
  and commands us to wash our hands.
\end{HgEnglish}

\vfill

\hchapter{רחצה}

\begin{HgEnglish}
It is customary to refrain from conversation from now until the blessing over
the Matzah, so that they can be spoken as a single blessing.
\end{HgEnglish}

\HgFill

\settowidth{\versewidth}{\vin Who shall ascend into the mountain of HaShem?}
\begin{verse}[\versewidth]
  The earth is HaShem's, and the fulness thereof; \\
  \vin the world, and they that dwell therein. \\
  For he hath founded it upon the seas, \\
  \vin and established it upon the floods. \\
  Who shall ascend into the mountain of HaShem? \\
  \vin and who shall stand in his holy place? \\
  He that hath clean hands, and a pure heart \\
\end{verse}
\HgSource{Psalm XXIV:1--5 \JSrc}
% JPS

\HgFill

\settowidth{\versewidth}{Would gush, flush, green these mountains and these
valleys}
\begin{verse}[\versewidth]
  \HgEllipsis

  It is our sorrow. Shall it melt? Ah, water \\
  Would gush, flush, green these mountains and these valleys, \\
  And we rebuild our cities, not dream of islands.
\end{verse}

\HgSource{W.~H.~Auden, [Hearing of harvests rotting in the valleys]}

\HgFill

\begin{HgEnglish}
  My soul melteth away for heaviness; sustain me according unto Thy word.

  \HgSource{Psalm CIX:28 \SSrc}
  % JPS
\end{HgEnglish}

\HgFill

\settowidth{\versewidth}{\vin that the bones which Thou hast crushed may
rejoice}
\begin{verse}[\versewidth]
  Wash me thoroughly from mine iniquity, \\
  \vin and cleanse me from my sin. \HgEllipsis \\
  Purge me with hyssop, and I shall be clean; \\
  \vin wash me, and I shall be whiter than snow. \\
  Make me to hear joy and gladness; \\
  \vin that the bones which Thou hast crushed may rejoice.
\end{verse}
\HgSource{Psalm LI:4,9--10 \SSrc}
% JPS

\chapter*{Motzi: Blessing the bread \\ Matzah: Blessing the matzah}

\vfill

\HgInst{Hold all three pieces of matzah, and recite:}

\begin{HgHebrew}
בָּרוּךְ אַתָּה יי אֱלֹהֵינוּ מֶלֶךְ הָעוֹלָם
\\
הַמּוֹצִיא לֶחֶם מִן הָאָרֶץ.
\end{HgHebrew}

\begin{HgTranslit}
  Baru\d{h} ata Adonai, eloheinu mele\d{h} ha'olam, \\
  hamotzi le\d{h}em min ha'aretz.
\end{HgTranslit}

\begin{HgEnglish}
  Blessed are you, Lord our God, ruler of the universe, \\
  who brings forth bread from the earth.\\
\end{HgEnglish}

\HgFill

\HgInst{Break the top and middle matzah into pieces, and distribute them around
the table. Continuing to hold only those pieces, read:}

\begin{HgHebrew}
  בָּרוּךְ אַתָּה יי אֱלֹהֵינוּ מֶלֶךְ הָעוֹלָם,
  \\
  אֲשֶׁר קִדְּשָנוּ בְּמִצְוֹתָיו
  \phantom{וְצִוָּנוּ עַל אֲכִילַת מַצָּה.}
  \\
  וְצִוָּנוּ עַל אֲכִילַת מַצָּה.
\end{HgHebrew}

\begin{HgTranslit}
  Baru\d{h} ata Adonai, eloheinu mele\d{h} ha'olam, \\
  asher kidshanu b'mitzvotav, \\
  v'tzivanu al a\d{h}ilat matzah.
\end{HgTranslit}

\begin{HgEnglish}
  Blessed are you, Lord our God, ruler of the universe, \\
  who sanctifies us with your commandments, \\
  and commands us to eat matzah.
\end{HgEnglish}

\vfill

\hchapter{מוציא\\ מצה }

\begin{HgEnglish}
  It is said that Rabbi Levi Yitz\d{h}ak of Berditchev discovered that the young
  girls who kneaded the dough for matzah worked from early in the morning until
  late at night. Every 17 minutes, the workers had to scrub their hands, to be
  sure that no particle of yeast remained on their fingers that might leaven the
  next batch of dough; by the end of a shift, their hands were chapped and
  bloody.  The Rabbi pronounced the matzah treyf because it was produced through
  oshek (the oppression of workers), and said: ``They accuse us of making our
  matzah with the blood of Christians, but in truth we make it from the blood of
  our own people.'' \SSrc
\end{HgEnglish}

\HgFill

\begin{HgEnglish}
  It is a mitzvah that the Passover matzot be made specifically for the purpose
  of fulfilling the obligation to eat matzot on Passover, as the verse (Exodus,
  12:17) states: You shall guard the matzot. The matzah must be guarded to
  ensure that it does not become chametz; matzah which was not guarded may not
  be used on Passover.

  How do we ``guard'' the matzah used on Passover? 

  From the time that the wheat is taken to the mill to be ground into flour, it
  is kept under careful supervision to make sure that it does not come into
  contact with water or other moisture. We check it to ensure that the wheat is
  clean and whole.  

  The grinding, packing, and transporting of the wheat from the mill to the
  bakery is done under close supervision, to ensure that the ground wheat does
  not come into contact with water, and all of the utensils used for processing
  the wheat must be clean and dry.  

  The wheat is not baked on the same day on which it was ground, for it is still
  warm from the processing and would therefore ferment and become chametz more
  rapidly. It is watched while it is being kneaded to ensure that this is not
  done near an oven or open window which is exposed to the sun, lest the dough
  become hot and ferment quickly.  
  \HgEllipsis

  \HgSource{Chabad}
\end{HgEnglish}


\HgFill

\begin{HgEnglish}
  It is a Torah mitzvah to eat matzah on Seder night.  Jewish law defines an act
  of ``eating'' as swallowing a kezayit within two to four minutes (kiday
  achilat pras). If this is difficult, you may sip some water while eating. At
  the very least, the matzah must be consumed within nine minutes.  The time
  begins not with the first bite, but with the first swallow.  Therefore, you
  can gain some extra time by chewing up some matzah before taking the first
  swallow.  A kezayit is approximately 45-50 cc, which is roughly two thirds of
  a square matzah, or one half of the hand-made round matzah.

  \HgSource{Aish HaTorah}
\end{HgEnglish}

\chapter*{Maror: Blessing the bitter herb \\
         Korei\d{h}: Hillel's sandwich}
\vfill

\HgInst{Take a spoonful of maror. Dip it in the \d{h}aroset, then shake the
\d{h}aroset off. Now recite:}

\begin{HgHebrew}
  בָּרוּךְ אַתָּה יי אֱלֹהֵינוּ מֶלֶךְ הָעוֹלָם, 
  \\
  אֲשֶׁר קִדְּשָנוּ בְּמִצְוֹתָיו וְצִוָּנוּ 
  \phantom{עַל אֲכִילַת מָרוֹר.}
  \\
  עַל אֲכִילַת מָרוֹר.
\end{HgHebrew}

\begin{HgTranslit}
  Baru\d{h} ata Adonai, eloheinu mele\d{h} ha'olam, \\
  asher kidshanu b'mitzvotav, \\
  v'tzivanu al a\d{h}ilat matzah.
\end{HgTranslit}

\begin{HgEnglish}
  Blessed are you, Lord our God, ruler of the universe, \\
  who sanctifies us with your commandments, \\
  and commands us to eat maror.
\end{HgEnglish}

\HgFill

\HgInst{Distribute the third matzah. Take a spoonful of maror and a spoonful of
  \d{h}aroset, and place them between two pieces of matzah in a sandwich. Say:}

\HgEnglish{Thus did Hillel, in the days of the Temple, combine the
  Passover lamb, the matzah and the maror and eat them together, as it is said:
``They shall eat it with matzah and bitter herbs.''}

\vfill

%\HgInst{The meal is served! It is permitted to drink wine between the second and
%third cup.}

\hchapter{מרור \\ כורך }

\begin{HgEnglish}
Using mortar and bricks, the Jewish slaves, who were foreigners, the children of
\linebreak
refugees from famine in Canaan, built the pyramids. The charoset reminds us of
the mortar, a symbol of unrewarded toil. Tonight we eat charoset to remember all
exploitation of immigrant labor. Just as many Jews in this country once worked
in sweatshops, immigrants continue to work long hours for poverty wages. We
remember how our ancestors' work enriched the lives of the Egyptians, and
challenge ourselves to think about the ways that many of us currently benefit
from exploited labor. Immigrants plant, pick, and process our food; sew our
clothes; take care of our children; and clean our homes and offices. For Jews,
the descendants of slave laborers who built the pyramids, such profit should
never be sweet. Instead, we take the sweetness of charoset as a symbol of
resistance and the possibility of liberation for all.

\HgSource{Jews for Racial \& Economic Justice}
\end{HgEnglish}

\HgFill

\begin{HgEnglish}
  The bitter maror stimulates our sense; let us use it as a stimulus to action,
  to remind us that struggle is better than complicit acceptance of injustice.
  We taste the bitter herbs and recognize the bitter consequences of
  exploitation, the loss of lives and the waste of the powerful potential of all
  people. In memory not only of the Temple built by the Jews in Jerusalem that
  was destroyed first by the Babylonians and then by the Romans, but also in
  memory of all cultural artifacts destroyed cruelly in wars, we do as Rabbi
  Hillel did in the times of the Temple.

  {\itshape Blessed are you, who grated this bitter herb and put it in a jar so
  that we didn’t cry bitterly grating it ourselves.} \SSrc
\end{HgEnglish}

\HgFill

\begin{HgEnglish}
  We must understand these two forces, that of servitude and that of freedom,
  not as separate forces which operate independently, each with its own role,
  but as two joined, complementary forces\ldots

  The overall goal will be realized only with the knowledge and recognition that
  these two forces are not contradictory, but are joined in creating the world’s
  ultimate Freedom, the honor and the powerful desirability of which is not
  revealed unless it is crowned by exalted Servitude, servitude before the King
  of Glory, which is also the ultimate freedom.

  Therefore, the perfect form of Freedom comes when it is bound together with
  Servitude, such that a person will find within himself the total control which
  is suitable for a truly free person, who reigns as well upon the greatest of
  forces, which is the force of freedom itself.

  \HgSource{R. Yitzchak Kook, ``Haggadah shel Pesach''}
\end{HgEnglish}


\chapter*{Shul\d{h}an Orei\d{h}: Eating}
\vfill
\begin{center}
  \fontsize{50pt}{50pt}
  \selectfont
  The
  meal
\end{center}
\vfill

\hchapter{שולחן עורך}
\vfill
\begin{center}
  \fontsize{50pt}{50pt}
  \selectfont
  is 
  served.
\end{center}
\vfill

\chapter*{Tzafun: The hidden}

\vfill

\HgInst{If the afikoman has been hidden, find it.}\\

\HgInst{If the afikoman has been ransomed, pay for it.}\\

\HgInst{Once you have the afikoman back, share it among all the members of the
group.  This should be the last thing you eat tonight.}

\vfill

\hchapter{צפון}

\begin{HgEnglish}
In most families, it is customary to hide the matzah during {\bfseries
ya\d{h}atz}. At this point in the evening the children search for the matzah,
and a prize is awarded to the one who finds it.

Traditionally, it's even less straightforward. The seder cannot proceed until
the afikoman has been eaten, and so the child who holds it finds herself with an
excelleng bargaining token. Rather than simply being given a prize, she uses the
matzah to extort a gift of her choosing from the adults.
\end{HgEnglish}

\HgFill

\begin{HgEnglish}
  As mentioned before, the structure of the Passover seder appears to be
  modeled, at least in part, after a Greek symposium. At this point in the
  evening, the Greeks would begin the {\itshape epikomion}---after-dinner
  entertainment featuring drunken revelry, flute music and group sex.
\end{HgEnglish}

\HgFill

\begin{HgEnglish}
``If some of them fell asleep, they may eat [the afikoman when they awake]; if
all of them fell asleep, they must not eat.''

In the latter case they have a ceased to think about the Paschal lamb; when they
awake it is as though they would eat in two different places, sleep breaking the
continuity of action and place, and thus it is forbidden.

\HgSource{Tractate Pesa\d{h}im\footnote{Tr. Soncino}}\\
\end{HgEnglish}

\HgFill

\begin{HgEnglish}
The Afikomen is hidden away during Yachatz (division ceremony) at the beginning
of the Seder. Many families have the custom to allow the children to steal the
Afikomen. If we are trying to teach our children about Torah, how can we teach
them to steal?! The Afikomen represents the future redemption which is hidden
from us. Matzah, which must be eaten only after eating an appetizer to make us
hungry, represents a passion for truth. Eliyahu HaNavi, whom we symbolically
welcome with a fifth cup of wine on Seder-night, ``will return the heart of the
parents to the children and the children to the parents.'' The ``gap'' that
prevents one generation from relating to a previous one is our biggest problem.
When a generation takes the potential they have been given, and misappropriates
it by not applying it to Torah which is the one thing that can help us bridge
the gap between all past generations, they are stealing our future hope. We want
our children to steal the Afikomen instead; they should crave the ``quest'' for
Torah, represented by the matzah of the Afikomen, so that our final hidden
redemption can be revealed. 

\HgSource{R. Uziel Milevsky}
\end{HgEnglish}

\chapter*{Barei\d{h}: Blessing after the meal \\ {\LARGE The Grace After Meals}}

\vfill

\HgInst{Pour the third cup of wine.}

\HgInst{(If including the Song of Ascents, pick a translation from the facing
page.)}

\HgInst{Read:}

\begin{HgHebrew}
  בָּרוּךְ אַתָּה יי אֱלֹהֵינוּ מֶלֶךְ הָעוֹלָם 
  \\
  הַזָן אֶת הָעוֹלָם כֻּלּוֹ בְּטוּבוֹ בְּחֵן בְּחֶסֶד וּבְרַחֲמִים
  \\
  הוּא נוֹתֵן לֶחֶם לְכָל בָּשָׂר כִּי לְעוֹלָם חַסְדוֹ. 
  \\
  וּבְטוּבוֹ הַגָדוֹל תָּמִיד לֹא חָסַר לָנוּ, 
  \\
  וְאַל יֶחְסַר לָנוּ מָזוֹן לְעוֹלָם וָעֶד. 
  \\
  בַּעֲבוּר שְׁמוֹ הַגָּדוֹל, כִּי הוּא אֵל זָן וּמְפַרְנֵס לַכֹּל 
  \\
  וּמֵטִיב לַכֹּל, וּמֵכִין מָזוֹן לְכָל בְּרִיּוֹתָיו אֲשֶׁר בָּרָא. 
  \\
  בָּרוּךְ אַתָּה יי הַזָן אֶת הַכֹּל.
\end{HgHebrew}

\begin{HgTranslit}
  Baru\d{h} ata Adonai, eloheinu mele\d{h} ha'olam \\
  hazan et ha'olam kulo b'tuvo b\d{h}ein b'\d{h}esed u'vra\d{h}amim. \\
  Hu notein le\d{h}em l'\d{h}ol basar ki l'olam \d{h}asdo. \\
  U'vtuvo hagadol tamid lo \d{h}asar lanu, \\
  v'al ye\d{h}sar lanu mazon l'olam va'ed. \\
  Ba'avur shemo hagadol, ki hu elzan um'farnes lakol \\
  umetiv lakol, ume\d{h}in mazon, l'\d{h}ol b'riotav asher bara. \\
  Baru\d{h} ata Adonai, hazan et hakol.
\end{HgTranslit}
\begin{HgEnglish}
  Blessed are you, Lord our God, ruler of the universe \\
  who feeds the world with goodness, kindness and mercy. \\
  He gives food to all the world, for his kindness is everlasting. \\
  Through his great goodness we have not lacked food; \\
  and may we never lack it until the end of time, \\
  for his name's sake. For he sustains all, \\
  does good to all, and provides food to all the creatures he created. \\
  Blessed are you, Lord our God, who feeds all.
\end{HgEnglish}

\vfill

\hchapter{ברך}

\noindent
\begin{minipage}{.47\textwidth}
  \settowidth{\versewidth}{When God returned the exiles to Zion}
  \begin{verse}[\versewidth]
    A song of ascents: \\
    When God returned the exiles to Zion, \\
    \vin we were like dreamers. \\
    Then our mouths were filled with laughter \\
    \vin and our tongues with song; \\
    then it was said among the nations: \\
    \vin ``God has done great things for them.'' \\
    God has done great things for us, \\
    \vin and we rejoiced. \\
    Return us, God, from our exile, \\
    \vin like streams in the desert. \\
    The ones who sow in tears \\
    \vin will reap in joy; \\
    the ones who go out weeping, \\
    \vin carrying sacks of seeds, \\
    will return in joy, \\
    \vin bearing sheaves of wheat.
    %\HgSource{Mine, help from Chabad and Hillel}

    \HgSource{Psalm CXXVI}
  \end{verse}
\end{minipage}
\hfill
\begin{minipage}{.47\textwidth}
  \settowidth{\versewidth}{Our home looked as it looks in dreams:}
  \begin{verse}[\versewidth]
    A song of ascents: \\
    When we returned from far away \\
    Our home looked as it looks in dreams: \\
    The sun shines, gates swing \\
    Open of themselves, and someone \\
    Sings a song we had forgotten \\
    As we now remember laughter. \\
    Then strangers said, Great things \\
    Were done for them.  \\
    The Lord \\
    Did great things for us then. A good. \\
    But you must do great things again, \\
    Because we live with heaviness \\
    And twist and scatter like a river \\
    Delta bogged in marsh and reeds. \\
    We started sadly so we'd end up \\
    Smiling, for anyone begins, sows \\
    Seed with tears to reap his own, \\
    The happy harvest, no? 

    \HgSource{Psalm CXXVI\footnotemark[1]}
  \end{verse}
\end{minipage}\\[1em]
\footnotetext[1]{Tr. Wieder}

\HgFill
\begin{HgEnglish}
  This is the part where we say thank you. Traditionally, we say thank you to
  God; here we will instead thank those who resisted oppression. They have
  sustained us, both spiritually and materially, so that we too can come to
  resist and fight back against the forces oppress us.

  It was at Passover time in 1943 that the Warsaw Ghetto uprising took place. On
  that day, April 19th, began the revolt against the Nazis who had come into the
  Ghetto of Warsaw to complete the deportation of the remaining Jews. 40,000
  civilians were led by the Jewish Fighting Organization, several hundred young
  women and men armed with ancient guns and home-made Molotov cocktails.
  Confined in a small area of the Ghetto, they were unable to maneuver beyond a
  few city blocks. The leadership perished in the bunker at 18 Mila Street on
  May 8th; no one surrendered. For weeks thereafter, small groups of resistance
  fighters emerged to battle the Nazis from behind rubble and wreckage.
  Resistance continued until the Nazis burned the entire Ghetto to the ground in
  September 1943---six months after the uprising began. \SSrc
\end{HgEnglish}

\chapter*{Barei\d{h}: Blessing after the meal \\ {\LARGE The third cup and
Elijah's cup}}

\vfill

\begin{HgHebrew} 
  בָּרוּךְ אַתָּה יי אֱלֹהֵינוּ מֶלֶךְ הָעוֹלָם 
  \\
  בּוֹרֵא פְּרִי הַגָפֶן.
\end{HgHebrew}

\begin{HgTranslit}
  Baru\d{h} ata Adonai, eloheinu mele\d{h} ha'olam,
  borei p'ri hagafen.
\end{HgTranslit}
\begin{HgEnglish}
  Blessed are you, Lord our God, ruler of the universe,
  who creates the fruit of the vine.
\end{HgEnglish}
\vspace{-1em}

\HgInst{Drink the third cup, and pour the fourth. Fill a cup for Elijah as well.
  \\ Then open the door, and read:}

\begin{HgEnglish}
  Pour out thy wrath upon the heathen that have not known thee,  \\
  And upon the kingdoms that have not called upon thy name.  \\
  For they have devoured Jacob,  \\
  And laid waste his dwelling-place. 
  \HgSource{Psalm 79:6-7} \\[1em]
  Pour out thine indignation upon them, \\
  And let thy wrathful anger take hold of them. 
  \HgSource{Psalm 69:24} \\[1em]
  Persecute and destroy them in anger \\
  From under the heavens of the Lord. 
  \HgSource{Lamentations 3:66}
  \footnote{Tr. Webster}
  % Red Sea Haggadah
\end{HgEnglish}

\HgInst{Now sing:}

\begin{HgHebrew}
  אֵלִיָהוּ הַנָבִיא, אֵלִיָהוּ הַתִּשְׁבִּי, אֵלִיָהוּ הַגִלְעָדִי 
  \\
  בִּמְהֵרָה יָבוֹא אֵלֵינוּ עִם מָשִׁיחַ בֶּן דָוִד.
\end{HgHebrew}

\begin{HgTranslit}
  Eliyahu hanavi, Eliyahu hatishbi, Eliyahu hagiladi. \\
  Bimheira v'yameinu, yavo eileinu, im mashiach ben David.
\end{HgTranslit}
\vspace{-1em}

\begin{HgEnglish}
  Elijah the prophet, Elijah the Tishbite, Elijah of Gilead. \\
  Quickly, in our day, come to us with David's son the messiah.
\end{HgEnglish}

\vfill

\hchapter{ברך \\ {\LARGE \strut}}

\begin{HgEnglish}
  \HgHL{and I will redeem you with an outstretched arm, and with great
  judgments.}

  \vspace{-2em}
  \HgSource{Exodus 6:6 (JPS)}
\end{HgEnglish}

\HgFill

\begin{HgEnglish}
   In the diaspora, this craving for revenge was both understandable and
   ineffective. But the founding of the state of Israel has changed the
   situation completely. In Israel, Jews are far from being defenseless. We
   don’t have to rely on God to take revenge for the evils done unto us, past or
   present, real or imagined. We can pour out our wrath ourselves, on our
   neighbours, the Palestinians and other Arabs, on our minorities, on our
   victims.

   That is the real danger of the Haggadah, as I see it. It was written by and
   for helpless Jews living in perpetual danger. It raised their spirits once a
   year, when they felt safe for a moment, protected by their God, surrounded by
   their families

   %Taken out of this context and applied to a new, completely different
   %situation, it can set us on an evil course. Telling ourselves that everybody
   %is out to destroy us, yesterday and most certainly tomorrow, we consider the
   %grandiloquent bombast of an Iranian bigmouth as a living proof of the
   %validity of the old maxim. They are out to kill us, so we must – according to
   %another ancient Jewish injunction – kill them first.

   \HgEllipsis{} So, on this Seder evening, let our feelings be guided by the
   noble, inspiring part of the Haggadah, the part about the slaves who rose up
   against tyranny and took their fate in their own hands---and not the part
   about pouring out our wrath.

   \HgSource{Uri Avnery}
\end{HgEnglish}

\HgFill

\begin{HgEnglish}
Three thousand years ago, a farmer arose in the Middle East who challenged the
ruling elite. In his passionate advocacy for common people, Elijah sparked a
movement and created a legend which would inspire generations to come.

Elijah declared that he would return once each generation in the guise of
someone poor or oppressed, coming to people's doors to see how he would be
treated. Thus would he know whether or not humanity had become ready to
participate in the dawn of the Messianic age. He is said to visit every seder,
and sip there from his cup of wine.

\HgSource{R. Rachel Barenblat, ``The Velveteen Rabbi's Haggadah''}
\end{HgEnglish}


\HgFill

\begin{HgEnglish}
  Today we know that there is no Messiah coming; we must be our own Messiahs,
  bringing our own redemption and liberation, just as the Jews brought
  themselves out of Egypt and into freedom. So while we continue to open the
  door for any who care to come in, Elijah’s cup is to be drunk by all who wish
  to at the Seder tonight; for we must sustain ourselves in our ongoing
  struggles. \SSrc
\end{HgEnglish}


\chapter*{Hallel: Songs of praise}

\HgInst{Recite:}

\begin{HgHebrew}\
  יִשְׁתַּבַּח שִׁמְךָ לַָעַד מַלְכֵּנוּ,  \\
  הָאֵל הַמֶלֶךְ הַגָּדוֹל וְהַקָּדוֹשׁ בַּשָּׁמַיִם וּבַָאָרֶץ, \\
  כִּי לְךָ נָאֶה, יי אֱלֹהֵינוּ וֵאלֹהֵי אֲבוֹתֵינוּ ,  \\
  שִׁיר וּשְׁבָחָה, הַלֵּל וְזִמְרָה, עֹז וּמֶמְשָׁלָה, נֶצַח,  \\
  גְּדֻלָּה וּגְבוּרָה, תְּהִלָה וְתִפְאֶרֶת, קְדֻשָּׁה וּמַלְכוּת, \\
  בְּרָכוֹת וְהוֹדָאוֹת מֵעַתָּה וְעַד עוֹלָם.  \\
  בָּרוּךְ אַתָּה יי , אֵל מֶלֶךְ גָּדוֹל בַּתִּשְׁבָּחוֹת,  \\
  אֵל הַהוֹדָאוֹת, אֲדוֹן הַנִפְלָאוֹת, הַבּוֹחֵר בְּשִׁירֵי זִמְרָה,  \\
  מֶלֶךְ אֵל חֵי הָעוֹלָמִים. 
\end{HgHebrew}

\begin{HgTranslit}
  Yishtaba\d{h} shim\d{h}a l'ad malkeinu \\
  ha'el hamele\d{h} hagadol v'hakadosh bashamayim u'v'aretz \\
  ki l'\d{h}a na'eh, Adonai eloheinu v'elohei avoteinu \\
  shir u'sh'va\d{h}a, hallel u'v'zimrah, oz u'memshalah, netzah, \\
  g'dulah u'g'vurah, t'hilah v'tiferet, k'dushah umal\d{h}ut, \\
  bra\d{h}ot v'hoda'ot me'atah v'ad olam. \\
  Baru\d{h} ata Adonai, El mele\d{h} gadol batishba\d{h}ot. \\
  El hahoda'ot, Adon hanifla'ot, Habo\d{h}er bshirei zimrah \\
  Mele\d{h} el \d{h}ai haolamim.
\end{HgTranslit}

\begin{HgEnglish}
  Let your name be praised forever \\
  God, great King, holy in heaven and earth, \\
  For yours, Lord our God and God of our ancestors, \\
  is song and praise, celebration and psalm, power and dominion, and victory, \\
  greatness and strength, worship and glory, holiness and splendor, \\
  blessing and thanks, now and forever. \\
  Blessed are you, God, god and great King, \\
  God of thanksgiving, Master of wonders, creator of souls, delighter in song, \\
  Lord and life of worlds.
\end{HgEnglish}

\HgInst{One more time:}
\vspace{-1em}

\begin{HgHebrew} 
  בָּרוּךְ אַתָּה יי אֱלֹהֵינוּ מֶלֶךְ הָעוֹלָם בּוֹרֵא פְּרִי הַגָפֶן.
\end{HgHebrew}
%\vspace{-1em}

\begin{HgTranslit}
  Baru\d{h} ata Adonai, eloheinu mele\d{h} ha'olam,
  borei p'ri hagafen.
\end{HgTranslit}
\vspace{-2em}

\begin{HgEnglish}
  Blessed are you, Lord our God, ruler of the universe,
  who creates the fruit of the vine.
\end{HgEnglish}
\vspace{-1em}

\HgInst{Drink the fourth cup.}

\hchapter{הלל}


\noindent
\begin{minipage}{.36\textwidth}
  \vspace{-3em}

\settowidth{\versewidth}{\vin in thanking You, Lord God-of-us}
\begin{verse}[\versewidth]
Were it our mouths were filled \\
\vin with a singing like the sea, \\
Our tongues awash with song, \\
\vin as waves-countless, \\
And our lips to lauding, \\
\vin as the skies are wide, \\
And our eyes illumined \\
\vin like the sun and the moon, \\
And our hands spread-out \\
\vin like the eagles of heaven, \\
And our feet as fleet as fawns, \\
Still, we would not suffice \\
\vin in thanking You, Lord God-of-us
\HgEllipsis 
\end{verse}
\HgSource{Nishmat Kol \d{H}ai\footnotemark[1]}

\vspace{3em}

\begin{HgEnglish}
  \HgHL{and I will take you to Me for a people, and I will be to you a God.}

  \vspace{-1em}
  \HgSource{Exodus 6:6 \JSrc}
\end{HgEnglish}

\end{minipage}
\hfill
\begin{minipage}{.54\textwidth}
\settowidth{\versewidth}{A truth should come out plain and make good sense}
\begin{verse}[\versewidth]
  For once when I cried out somebody listened: \\
  The Lord took my complaint for a song \\
  Belted at the top of my lungs \\
  In a shower of troubles, good  \\
  Even if off key. Surrounded by death \\
  Loving liars, by fires banked inside \\
  My nature, I stumbled over simple things: \\
  A shoe untied, the pillow never smooth, \\
  A night cough, hum of strangers' tires.

  The gift has been given. So, low but alive, \\
  I said what I believed: that greed succeeds \\
  Where grace can not, that one idea can kill \\
  A world of simple pleasures, cup and spoon. \\
  Stir them. Don't speak quickly, savor \\
  The hot cider, candied ginger on the tongue,  \\
  Heavy cloud shrugged off my shoulders. \\
  I promise to be more than one of those \\
  (What, still alive? He lived? He died?) who never \\
  Show their heart or read the lips of mumblers \\
  In the public record: I call upon the Lord, \\
  Am called upon, to praise in easy words: \\
  A truth should come out plain and make good sense, \\
  So truth will find a friendly audience.
\end{verse}
\HgSource{Laurance Wieder}\\
\end{minipage}
\footnotetext[1]{Tr. Englander}

%\HgFill
\vspace{-2em}
\settowidth{\versewidth}{Landscape plotted and pieced---fold, fallow and plough;}
\begin{verse}[\versewidth]
Glory be to God for dappled things--- \\
\vin  For skies of couple-colour as a brinded cow; \\
\vin \vin    For rose-moles all in stipple upon trout that swim; \\
Fresh-firecoal chestnut-falls; finches’ wings; \\
\vin  Landscape plotted and pieced---fold, fallow, and plough; \\
\vin \vin    And \'{a}ll tr\'{a}des, their gear and tackle and trim. 
 
All things counter, original, spare, strange; \\
\vin  Whatever is fickle, freckled (who knows how?) \\
\vin \vin    With swift, slow; sweet, sour; adazzle, dim; \\
He fathers-forth whose beauty is past change: \\
\vin \vin \vin                  Praise him.\\
\HgSource{Gerard Manley Hopkins}
\end{verse}

\chapter*{Nirtzah: Conclusion}

\vfill

\begin{HgHebrew}
  לְשָׁנָה הַבָּאָה בִּירוּשַָׁלָיִם!
  \vspace{-1em}
\end{HgHebrew}

\begin{HgTranslit}
  \HgHL{L'shanah haba'ah b'yerushalayim!}
  \vspace{-3em}
\end{HgTranslit}

\begin{HgEnglish}
  \HgHL{Next year in Jerusalem!}
\end{HgEnglish}

\HgInst{Sing!}

\noindent
\begin{minipage}{.45\textwidth}
\begin{HgEnglish}
Who knows one oh? \\
{\itshape E\d{h}ad mi yodea?} \\
I'll sing you one, oh: \\
One is God in heaven and earth \\
and evermore shall be so. \\
\HgEllipsis \\
Thirteen are the attributes of God \\
Twelve are the tribes of Israel \\
Eleven are the stars in Joseph's dream \\
Ten are the ten commandments \\
Nine are the months of pregnancy \\
Eight are the days of the {\itshape b'rit milah} \\
Seven are the days of creation \\
Six are the books of the Mishnah \\
Five are the books of the Torah \\
Four are the Matriarchs \\
Three are the Patriarchs \\
Two are the tables of the covenant \\
One is God in heaven and earth \\
and evermore shall be so.
\end{HgEnglish}
\end{minipage}
\hfill
\begin{minipage}{.45\textwidth}
\begin{HgEnglish}
{\itshape \d{H}ad gadya, \d{h}ad gadya}. \\
\strut \hfill (An only kid, an only kid) \\
\HgEllipsis \\
Then came the Holy One, blessed be \\
\vin he, \\
and destroyed the angel of death, \\
who slew the butcher, \\
that slaughtered the ox, \\
that drank the water, \\
that quenched the fire, \\
that burned the stick, \\
that beat the dog, \\
that bit the cat, \\
that ate the kid \\
my father bought for two {\itshape zuzim}. \\
{\itshape \d{H}ad gadya, \d{h}ad gadya}.
\end{HgEnglish}
\end{minipage}

\vfill

\hchapter{נירצה}

\begin{multicols}{2}
\noindent
\small
This is the year that squatters evict \\
\vin landlords, \\
gazing like admirals from the rail \\
of the roofdeck \\
or levitating hands in praise \\
of steam in the shower; \\
this is the year \\
that shawled refugees deport judges \\
who stare at the floor \\
and their swollen feet \\
as files are stamped \\
with their destination; \\
this is the year that police revolvers, \\
stove-hot, blister the fingers \\
of raging cops, \\
and nightsticks splinter \\
in their palms; \\
this is the year \\
that darkskinned men \\
lynched a century ago \\
return to sip coffee quietly \\
with the apologizing descendants \\
of their executioners. \\
 
\noindent
This is the year that those \\
who swim the border's undertow \\
and shiver in boxcars \\
are greeted with trumpets and drums \\
at the first railroad crossing \\
on the other side; \\
this is the year that the hands \\
pulling tomatoes from the vine \\
uproot the deed to the earth that sprouts \\
\vin the vine, \\
the hands canning tomatoes \\
are named in the will \\
that owns the bedlam of the cannery; \\
this is the year that the eyes \\
stinging from the poison that purifies \\ 
\vin toilets \\
awaken at last to the sight \\
of a rooster-loud hillside, \\
pilgrimage of immigrant birth; \\
this is the year that cockroaches \\
become extinct, that no doctor \\
finds a roach embedded \\
in the ear of an infant; \\
this is the year that the food stamps \\
of adolescent mothers \\
are auctioned like gold doubloons, \\
and no coin is given to buy machetes \\
for the next bouquet of severed heads \\
in coffee plantation country. \\
 
\noindent
%\settowidth{\versewidth}{if every rebellion begins with the idea}
%\begin{verse}[\versewidth]
If the abolition of slave-manacles \\
began as a vision of hands without \\
\vin manacles, \\
then this is the year; \\
if the shutdown of extermination camps \\
began as imagination of a land \\
without barbed wire or the \\
\vin crematorium, \\
then this is the year; \\
if every rebellion begins with the idea \\
that conquerors on horseback \\
are not many-legged gods, that they too \\
\vin drown \\
if plunged in the river, \\
then this is the year. \\
 
\noindent
So may every humiliated mouth, \\
teeth like desecrated headstones, \\
fill with the angels of bread.\\
%\end{verse}

\HgSource{Mart\'in Espada}
\end{multicols}

\HgFill

\begin{HgEnglish}
  \small
May we see the day when war and bloodshed cease---when a great peace will
embrace the whole world.  Then nation shall not threaten nation, and humankind
will not again know war. \HgEllipsis{}
%For all who live on earth shall realize we have not
%come into being to hate or destroy. 
%We have come into being to praise, to
%labor, and to love.  
% Bless the leaders of all nations with the power of compassion. 
Fulfill the promise conveyed in Torah: ``I will bring peace to the
land, and you shall lie down and no one shall terrify you.  I will rid the land
of vicious beasts, and it shall not be ravaged by war.'' Let love and justice
flow like a mighty stream.  Let peace fill the earth as the waters fill the sea.
And let us say: Amen. \HgSource{R. Nachman of Breslov}
\end{HgEnglish}

\chapter*{A brief history of this haggadah}

During my undergraduate years, the annual cobbling-together of a haggadah from
assorted prayer books, Maxwell House freebies and miscellaneous online snippets
grew into a March ritual of its own. The haggadah, once constructed and
wine-baptized, was invariably lost---and, in an echo of the pagan springtime
ceremony from which Passover likely derives, refashioned anew each year.  This
document is an attempt to stabilize the process.

Every haggadah is a palimpsest, but this one more than most. Its immediate
progenitor, which I refer to here as the ``Commune Haggadah'', was used in 2011
and 2012 (and possibly earlier) by an eponymous collective of Columbia students
who identified---some ironically and some with deadly seriousness---with various
worldwide socialist movements. The Commune Haggadah appears to have itself been
substantially plagiarized from an earlier ``Socialist Haggadah'' (with the full
cooperation of the latter's author), about which I have been able to determine
only that it was prepared by a friend-of-a-friend, one Eve Goodman, from an
intimidating list of scholarly sources between the years 2003 and 2010.

This is not a Marxist holiday, nor a Marxist haggadah. Nevertheless the basic
story of Passover is the liberation of the oppressed by overwhelming
historical forces, and the basic impulse of the Commune Haggadah was good. It,
and its predecessors, still constitute the spiritual (if no longer the textual)
core of the present document. I can only hope that, transplanted with its
keeper, the Haggadah you now hold will flourish as we have in Cambridge's
strange soil.

\section*{Sources}

All text prepared new for this version is 
%marked \CSrc, and is 
hereby released
into the public domain. The following sources have been instrumental in
assembling the remainder of this document:
\begin{itemize}
  \item {\bfseries Hebrew text} is traditional; my reference has been the Chabad
    Haggadah.\footnote{http://www.chabad.org/holidays/passover/pesach\_cdo/aid/1735/jewish/The-Haggadah.htm}
  \item {\bfseries Translations of biblical text} are from the (public-domain)
    1917 JPS
    Version,\footnote{http://en.wikisource.org/wiki/Bible\_(Jewish\_Publication\_Society\_1917)}
    and marked \JSrc.
  \item {\bfseries Some translations of traditional commentary} are from the
    Chabad Haggadah, and marked \LSrc.
  \item {\bfseries Alternative commentaries} from the Socialist Haggadah
    are marked \SSrc.
\end{itemize}
Other borrowings (the bulk of this haggadah) are noted in-place. All texts not
in the public domain are subject to the copyright of their respective owners,
and included here for educational purposes.
\end{document}
